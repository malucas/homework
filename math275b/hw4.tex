\documentclass[10pt]{article}

\usepackage{amsthm}
\usepackage{amsmath}
\usepackage{amsfonts}
\usepackage[margin=1in]{geometry}
\usepackage{titlesec}

\titleformat{\section}{\large\bfseries}
{\thesection}{0.5em}{}
\titleformat{\subsection}{\normalsize\itshape}
{\thesubsection}{0.5em}{}
\titlespacing*{\subsection}{0pt}{0pt}{0.1em}

\setlength{\parskip}{\baselineskip}
\setlength{\parindent}{0pt}

\title{Math 275B: Homework 4}
\author{Marcus Lucas}
\date{\vspace{-1cm}}

\newcommand{\var}{\text{var}}

\begin{document}

\maketitle

\subsection*{D.5.2.4}
\textit{Claim.} Let $T_y  = \inf\{n \ge 1: X_n = y\}$.
Show that:
\begin{equation*}
  p^n(x,y) = \sum_{m=1}^n P_x(T_y = m)p^{n-m}(y,y)
\end{equation*}

\begin{proof}
  Let $Y_m: \Omega_0 \to \mathbb{R}$
  define a family of random varibles for which $Y_m(w) = 1$ 
  if $w_{(n-m)} = y$ and $Y_m(w) = 0$ otherwise. Then,
  using $\theta_n$ to specify the left-shift of a sequence 
  $w$ by $n$ steps, we can use the strong Markov property
  to determine:
  \begin{equation*}
  \begin{aligned}
    p^n(x,y) & = E_x(Y_N \circ \theta_N; N \le n)
    = E_x(E_x(Y_N \circ \theta_N \vert \mathcal{F}_N); N \le n)
    = E_x(E_{X_N}Y_N; N \le n) \\ 
    & = E_x(p^{n-N}(y,y); N \le n) 
    = \sum_{m=1}^n p^{n-m}(y,y) E_x(; N = m)
    = \sum_{m=1}^n p^{n-m}(y,y) P_x(T_y = m).
  \end{aligned}
  \end{equation*}
\end{proof}

\subsection*{D.5.3.3}
\textit{Claim.} Show that in the Ehrenfest chain,
all states are recurrent.

\begin{proof}
  The set $S = \{1,2,\dots,r\}$ is obviously closed
  since probabilities $\rho_{ij} > 0$ are only defined
  for $i,j \in S$. Likewise $S$ is irreducible since for
  any $i,j \in S$ where $i<j$:
  \begin{equation*}
    \rho_{ij} \ge p^{j-i}(i,j) 
    = \prod_{k=0}^{j-i-1} p(i+k,i+k+1)
    = \prod_{k=0}^{j-i-1} \frac{r-k}{r} > 0.
  \end{equation*}
  Theorem 5.3.3 in Durrett tells us that because $S$ is
  finite, closed, and irreducible, all states in $S$ 
  are recurrent, proving the claim.
\end{proof}

\subsection*{D.5.5.2}
\textit{Claim.} Let $w_{xy} = P_x(T_y < T_x)$.
Show that $\mu_x(y) = w_{xy}/w_{yx}$.

\begin{proof}
  Here I'll instead show that $\mu_x(y) w_{yx} = w_{xy}$.
  Starting with the definition of $\mu_x(y)$ in
  theorem D.5.5.7:
  \begin{equation*}
  \begin{aligned}
    \mu_x(y) w_{yx} 
    & = \left( \sum_{n=0}^\infty P_x(X_n = y, T_x > n) \right) P_y(T_x < T_y)
    = \sum_{n=0}^\infty P_x(X_n = y, T_x > n) P_y(T_x < T_y) \\
    & = \sum_{n=0}^\infty P_x(X_n = y, T_x > n, X_m \ne y 
    \ \text{for} \ n < m < T_x).
  \end{aligned}
  \end{equation*}
  The last summand just describes the probablity that
  $X_n = y$ is the last occurance of $y$ (starting from $y$)
  before $X_m = x$ for some $m > n$. But this description 
  describes all the possible sequences $w \in \Omega_0$ for which
  $X_0 = x$ and $T_y < T_x$ are satisfied.
  Furthermore, each set $\{X_0 = x, X_n=y, X_m \ne y \ \text{for} \
  n < m < T_x \}$ is disjoint from the others for unique $n$,
  such that the sum of their probabilities gives 
  $P_x(T_y<T_x) = w_{xy}$.
\end{proof}

\subsection*{D.5.6.2}
\textit{Claim.} Show that if $S$ is finite and $p$ is irreducible
and aperiodic, then there is an $m$ that $p^m(x,y) > 0$ for all 
$x,y \in S$.

\begin{proof}
  $S$ is irreducible so for any $x,y \in S$, $\rho_{xy} > 0$.
  Theorem D.5.6.4 then tells us that $d_x = d_y$ for all
  $x,y \in S$. Since $S$ is also aperiodic, actually $d_x = 1$
  for all $x \in S$.
  
  Lemma then D.5.6.5 tells us that for each $x \in S$ there exists
  a positive integer $m_x$ such that for $m \ge m_x$, $p^m(x,x) > 0$.
  Since $S$ is finite, let $m_0 = \max_{x \in S} m_x$ such that
  $p^m(x,x) > 0$ for any $x \in S$ and $m > m_0$.

  Lastly, since $\rho(x,y)>0$ for each pair $x,y \in S$, 
  there exists some positive integers $n_{xy}$ such that 
  $p^{n_{xy}}(x,y) > 0$. Note that for such integers, 
  it must hold that $p^{n_{xy}+m}(x,y) 
  > p^{n_{xy}}(x,y)p^{m}(y,y) > 0$ for any $m \ge m_0$ as well. 
  Thus let $n_0 = m_0 + \max_{x,y \in S} n_{xy}$.
  Then the claim holds for any $x,y \in S$ and $m > n_0$.
\end{proof}

\end{document}