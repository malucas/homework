\documentclass[11pt,letter]{article}
\usepackage{fancyhdr}
\usepackage[]{amsmath}
\usepackage[margin=1in]{geometry}
\usepackage{amsfonts}
\usepackage{amssymb}
\pagestyle{fancy}
\fancyhead[LO]{MATH245B: HW \#2}
\fancyhead[CO]{Marcus Lucas}
\fancyhead[RO]{January 24, 2019}

\begin{document}
\begin{enumerate}
    \item \begin{enumerate}
        \item The set of balls defined by $\rho$ and $\rho'$ form bases for the topologies $\mathcal{T}$ and $\mathcal{T}'$ which they generate. Given an open set $U \in \mathcal{T}$, for every point $x \in U$ there exists an open ball $\mathcal{B}_\rho(x)$ containing $x$ which lies completely within $U$. But we can also generate another ball $\mathcal{B}_{\rho'}'(x)$ of possibly smaller radius which fits within $\mathcal{B}_\rho(x)$ and still contains $x$. Thus $\mathcal{T} \subseteq \mathcal{T}'$.
        
        The same argument can be made to show that $\mathcal{T}' \subseteq \mathcal{T}$ such that the topologies generated by $\rho$ and $\rho'$ are equivalent.

        \item $\rho(\mathbf{x},\mathbf{y})$ is clearly non-negative since each $\rho_n$ is a metric. Also $\rho(\mathbf{x},\mathbf{y}) = 0$ implies that $\rho_n(x_n,y_n) = 0$ for all $n \in \mathbb{N}$ such that $\mathbf{x}=\mathbf{y}$. Symmetry holds since
        \begin{align*}
            \rho(\mathbf{x},\mathbf{y}) = \text{max}\{\rho_n(x_n,y_n)/n\} = \text{max}\{\rho_n(y_n,x_n)/n\} = \rho(\mathbf{y},\mathbf{x})
        \end{align*}
        where we exploit the symmetry of each metric $\rho_n$. Finally we can demonstrate the triangle inequality as follows:
        \begin{align*}
            \rho(\mathbf{x},\mathbf{y}) + \rho(\mathbf{y},\mathbf{z})
             & = \text{max}\{\rho_n(x_n,y_n)/n\} + \text{max}\{\rho_n(y_n,z_n)/n\} \\
             & \ge \text{max}\{\rho_n(x_n,y_n)/n + \rho_n(y_n,z_n)/n\} \\
             & \ge \text{max}\{\rho_n(x_n,z_n)/n\} \\
             & = \rho(\mathbf{x},\mathbf{z})
        \end{align*}

        Take any ball $\mathcal{B}_\rho(\mathbf{x})$ of finite radius $r$ and let $N$ equal the smallest integer greater than or equal to $r^{-1}$. For all integers $n \ge N$, we will have that $\pi_n(\mathcal{B}_\rho(\mathbf{x})) = X_n$. Thus any such ball can be represented as a product of open sets $\prod_{n\in\mathbb{N}} U_n$ where $U_n \ne X_n$ for finitely many $n$. But this corresponds to the product topology generated by the $\rho_n$s.

        \item More generaly. If we have a product topology on a couuntably infinite product of metrizable spaces, i.e. $X = \prod_{n \in \mathbb{N}} X_n$, we can generate that same topology in two steps. First, replace each metric $\rho_n$ on $X_n$ by $\rho'(x,y) = \text{min}\{\rho(x,y),1\}$. As shown in part (a), each $\rho_n'$ generates the same topology on $X_n$ as the orinial metric $\rho_n$. Then apply the result from part(b) to create $\rho(\mathbf{x},\mathbf{y}) = \text{max}\{\rho_n'(x_n,y_n)/n: n \in \mathbb{N}\}$. As we've just shown, this metric will generate the desired product topology, meaning that it is metrizable.
    \end{enumerate}
        
    \item[4.20] If each $X_\alpha$ is first countable then every point $x_\alpha \in X_\alpha$ possesses a countable neighborhood. Given some $x \in X = \prod_{\alpha \in A} X_\alpha$, any open set $U$ containing $x$ has a finite number of components $U_\alpha \ne X_\alpha$. We can categorize all such open sets contain $x$ by the finite sequence $\langle \alpha_j \rangle$ of indices of such $U_{\alpha_j}$.
    
    Now the number of finite subsets of $\mathbb{N}$ (and thus any countable set $A$) is countable. Furthermore for any finite sequence $\langle \alpha_j \rangle$, we can form a countable neighborhood base around $x$ by taking the set $\{\prod_{\alpha_j \in \langle \alpha_j \rangle} \pi_{\alpha_j}^{-1}V_{\alpha_j}\}$ where $V_{\alpha_j}$ is any member of the countable neighborhood base for $\pi_{\alpha_j}(x)$. This set is a finite product of countable sets and so is countable. The sum of all these sets over all the finite sequences of $A$ then is also countable and forms a neighborhood basis for $x$ in the product topology.

    The same line of reasoning can be used to the second countable case. If each $X_\alpha$ has a countable basis $B_\alpha$, the collection of all sets of the form $\prod_{\alpha_j \in \langle \alpha_j \rangle} U_{\alpha_j}$ where $U_{\alpha_j} \in B_{\alpha_j}$ forms a countable basis for $X$.
    \newpage 
    \item[4.22] Each sequence $\langle f_n(x_0) \rangle$ is cauchy given the upper bound on the distance between successive points $f_n(x_0)$ and $f_m(x_0)$, where $m \ge n \ge N$ converges to zero as $N \rightarrow \infty$. Since $(\mathcal{Y},\rho)$ is complete, each sequence $\langle f_n(x_0) \rangle$ then has a limit point $f(x_0)$ whithin the space. Let $f(x)$ be the map definied by these limit points. Since $\text{sup}_{x\in X}(\rho(f_n(x),f_m(x))) \rightarrow 0$, it follows that $\text{sup}_{x\in X}(\rho(f_n(x),f(x))) \rightarrow 0$.
    
    If there is another function $g$ such that $\text{sup}_{x\in X}(\rho(f_n(x),g(x))) \rightarrow 0$ then we have that $\rho(f_N(x),f(x)) < \epsilon$ and $\rho(f_N(x), g(x)) < \epsilon$ from some $N \in \mathbb{N}$. But then $\rho(f(x),g(x)) < 2\epsilon$ and we can make $\epsilon$ arbitrarily small. Thus $f = g$ and the limiting function is unique.

    Take some neighborhood $V$ of $f(x_0)$ for any point $x_0 \in X$. Then there exists a ball $\mathcal{B}_{3\epsilon}(f(x_0)) \subseteq V$ containing $x_0$, and an $N$ such that $\text{sup}_{x\in X}(\rho(f_N(x),f(x))) < \epsilon$. Since $f_N$ is continuous, $U = f_N^{-1}(\mathcal{B}_{3\epsilon}(x_0))$ is open in $X$. Furthermore if $x \in U$ then
    \begin{align*}
        \rho(f(x_0),f(x)) \le \rho(f(x_0), f_N(x_0)) + \rho(f_N(x_0), f_N(x)) + \rho(f_N(x), f(x)) = 3\epsilon
    \end{align*}
    meaning that $f^{-1}(U)$ must be a neighborhood of $x_0$. Since $x_0$ was arbitrary, this means $f$ is continuous.

    \item[4.24] If $X$ is normal then the conclusions of Urysohn and Tietze follow immediately. Now assume the conculsion of Urysohn. For any disjoint closed subsets $A$ and $B$, there exists $f \in C(X,[0,1])$ such that $f=0$ on $A$ and $f=1$ on $B$. Since $f$ is continuous, $f^{-1}(-\infty,\frac{1}{2})$ and $f^{-1}(\frac{1}{2}, \infty)$ are disjoint open sets containing $A$ and $B$ respectively.
    
    Furthermore, if we assume the conclusion of Tietze's Theorem, then for disjoint closed $A$ and $B$ we can form $X' = A \cup B$ on which $A$ and $B$ are in fact open under the restricted topology. Thus the characteristic function $\chi_B$ is continuous on $X'$, implying the existence of some $f \in C(X,[0,1])$ which satisfies the conclusion of Urysohn's lemma. This in turn implies that $X$ is normal.
    
    \item[4.38] If $\mathcal{T}'$ is strictly stronger than $\mathcal{T}$, then it must be Hausdorff, since we can always use two open sets definted in both topologies to separate any points $x,y \in X$. Now assume $\mathcal{T}'$ is compact and let $V \in \mathcal{T}'\setminus \mathcal{T}$. $V^c$ is closed and thus compact. The identity map from $(X,\mathcal{T}')$ to $(X,\mathcal{T})$ is continuous and so $V^c$ must also be compact relative to $\mathcal{T}$. But since $\mathcal{T}$ is Hausdorff, this means that $V^c$ is closed in $\mathcal{T}$ and that $V \in \mathcal{T}$, violating our assumption. Thus $\mathcal{T}'$ cannot be compact.
    
    If $\mathcal{T}'$ is strictly weaker than $\mathcal{T}$, then we can demonstrate compactness by observing the identity map from $(X,\mathcal{T})$ to $(X,\mathcal{T}')$. It is continuous and so $\mathcal{T}'$ must be compact. If $\mathcal{T}'$, then we can apply the same line of reasoning as before to show that some $V \in \mathcal{T}\setminus \mathcal{T}'$ must actually be containted in $\mathcal{T}$. Thus $\mathcal{T}'$ cannot be Hausdorff.
    
    \item[4.43] Given the sequence $\langle a_n \rangle$, let $\langle a_{n_k} \rangle$ be any subsequence. We can define $x = \sum_{k \in \mathbb{N}} (1 + (-1)^k)2^{-(n_k + 1)}$, such that $\langle a_{n_k} \rangle$ becomes $(0,1,0,1,0,\dots)$. This sequence does not converge, meaning $\langle a_{n_k} \rangle$ is not pointwise convergent and $\langle a_n \rangle$ has no pointwise convergent subsequence.
\end{enumerate}
\end{document}
