\documentclass[11pt,letter]{article}
\usepackage{fancyhdr}
\usepackage[]{amsmath}
\usepackage[margin=1in]{geometry}
\usepackage{amsfonts}
\usepackage{amssymb}
\pagestyle{fancy}
\fancyhead[LO]{MATH245B: HW \#3}
\fancyhead[CO]{Marcus Lucas}
\fancyhead[RO]{January 31, 2019}

\begin{document}
\begin{enumerate}
    \item [4.56)]\begin{enumerate}
        \item For two values $t,s \in [0,\infty]$, if $s < t$ then
        \begin{align*}
            st + s < st + t \implies s(t+1) < t(s+1) \implies \frac{s}{s+1} < \frac{t}{t+1} \ \text{or} \ \Phi(s) < \Phi(t)
        \end{align*}
        Also
        \begin{align*}
            \Phi(t + s) = \frac{t+s}{t+s+1} = \frac{t}{t+s+1} + \frac{s}{t+s+1} < \frac{t}{t+1} + \frac{s}{s+1} = \Phi(t) + \Phi(s)
        \end{align*}
        
        \item $\Phi$ is monotic and so it has an inverse. This means that $\rho(x,y) \le r$ implies \\$\Phi \circ \rho(x,y) \le \Phi(r)$ and also $\Phi \circ \rho (x,y) < r$ implies $\rho(x,y) < \Phi^{-1}(r)$. Thus set $U$ open in $(Y,\rho)$ is also open under the topology generated by $\Phi \circ \rho$ since any open ball with respect to $\rho$ must include an open ball with respect to $\Phi$ and vice versa. This implies that the topologies generated by either metric are the same.
        
        \item Let $\gamma(f,g) = \text{sup}_{x\in X} \vert f(x) - g(x) \vert$. Then $(C(X),\gamma)$ is a metric space with the topology of uniform convergence. The previous result tells us that $\Phi \circ \gamma$ is also a metric on $C(X)$ with the same topology.
        
        \item We already know that each summand of $\rho$ is a metric. Thus $\rho$ is a metric since a positive sum of metrics is still a metric.
        
        For any open ball $B_r(f)$, each function $g \in B_r(f)$ must be contained within some ball $B_s(g)$ such that $B_s(g) \subset B_r(f)$. If we let $B'_{k,N}(g) = \{h: \text{sup}_{x \in \overline{U_n}}\vert h(x) - g(x) \vert < k^{-1} \}$ such that $\Phi(k^{-1} < \frac{s}{2})$ and $\sum_{n=N}^\infty 2^{-n} < \frac{s}{2}$ then it folows that
        \begin{align*}
            B'_{k,N}(g) \subset \{h: \sum_{n=1}^{N-1} 2^{-n} \Phi(\underset{x \in \overline{U_N}}{\text{sup}}\vert h(x) - g(x) \vert) < \frac{s}{2} \} \subset B_s(g).
        \end{align*}
        So $B_r(f)$ is open in the topology of uniform convergence of compact sets.

        Likewise, for any open ball $B'_{k,N}(f)$ in the topology of uniform convergence. For any $g \in B'_{k,N}(f)$ there exists an $s$ such that $\{h: \text{sup}_{x \in \overline{U_n}}\vert h(x) - g(x) \vert < s \} \subset B'_{k,N}$. But this implies that $B'_{k,N}(f)$ also contains $B_{2^{-N}\Phi(s)}$ since
        \begin{align*}
            B_{2^{-N}\Phi(s)} & = \{h: \sum_{n=1}^{N-1} 2^{-n} \Phi(\underset{x \in \overline{U_n}}{\text{sup}}\vert h(x) - g(x) \vert) < 2^{-n}\Phi(s) \} \\ 
            & \subset \{h: 2^{-N} \Phi(\underset{x \in \overline{U_N}}{\text{sup}}\vert h(x) - g(x) \vert) < 2^{-N} \Phi(s)\} \\
            & = \{h: \underset{x \in \overline{U_N}}{\text{sup}}\vert h(x) - g(x) \vert < s\} \subset B'_{k,N}
        \end{align*}
        So any $B'_{k,N}$ is open in the metric topology and thus the two topologies are equivalent.

        Lastly, a sequence of functions converges in the metric iff they converge in the topology of uniform convergence of compact sets (since it is equivalent to the metric topology). Since $X$ is LCH an thus every point $x \in X$ has a compact neighborhood. This implies that such a convergent sequence will converge uniformly on those compact neighborhoods, i.e. it will converge locally uniformly.
    \end{enumerate}

    \newpage

    \item [4.61)] \textbf{Restatement:} Let $X$ be a compact Hausdorff space and $Y$ be a metric space. If $\mathcal{F}$ is an equicontinuous, pointwise totally bounded subset of functions mapping $X$ to $Y$, then $\mathcal{F}$ is totally bounded in the uniform metric and $\mathcal{F}$ is compact in the space of all functions from $X$ to $Y$.
    
    The proof is almost the same as that for theorem 4.43. As in the original proof, once we have a finite set $x_1, \dots, x_n \in X$ for which $\cup_1^n U_x = X$, we need to find a finite set $\{z_n\} \subset Y$ that is $\frac{1}{4}\epsilon$-dense in $\{f(x_j):f\in\mathcal{F}, 1\le j \le n\}$. Here, we need pointwise total bounded-ness to be able to claim there is a finite covering of each set $ \{f(x_j):f\in\mathcal{F}\}$ with balls of radius $\frac{1}{4}\epsilon$, the centers of which we can use to form our finite set $\{z_n\}$.

    From here, the proof proceeds the same as for theorem 4.43.
    
    \item [4.63)] Any functions $K$ and $f$ have compact images since their domains are compact. Compact sets are bounded in $\mathbb{C}$ and so their images are also bounded. In particular, $\Vert f \Vert_u$ is finite. Also by continuity of $K$ we can pick a $\delta$ such that $\vert K(x_1,y) - K(x_2,y) \vert < \epsilon / \Vert f \Vert_u$ whenever $\vert x_1 - x_2 \vert < \delta$. This implies the following:
    \begin{align*}
        \vert Tf(x_1) - Tf(x_2) \vert &  = \int_0^1 K(x_1,y)f(y)\,dy - \int_0^1 K(x_2,y)f(y)\,dy \\
        & \le \int_0^1 \vert K(x_1,y) - K(x_2,y) \vert \Vert f \Vert_u \,dy \\
        & < \int_0^1 \frac{\epsilon}{\Vert f \Vert_u} \Vert f \Vert_u \,dy = \epsilon
    \end{align*}
    and so $Tf \in C([0,1])$.

    To show that $\mathcal{F} = \{Tf: \Vert f \Vert_u \le 1\}$ is equicontinuous note that for any point $x_1 \in [0,1]$, we can find a $\delta$ such that $\vert K(x_1,y) - K(x_2,y) \vert < \epsilon$ for all $x_2$ such that $\vert x_1 - x_2 \vert < \delta$. Since, we have $\Vert f \Vert_u \le 1$ this also means that $\vert K(x_1,y) - K(x_2,y) \vert < \epsilon/\Vert f \Vert_u$ for any $f \in \mathcal{F}$. Thus by the proof of continuity of a specific $Tf$ we have that $\mathcal{F}$ is equicontinuous at a point $x_1$. The choice of $x_1$ was arbitrary so $\mathcal{F}$ is actually just equicontinuous.

    Furthermore, since $\vert Tf(x) \vert \le \int_0^1 \vert  K(x,y) \vert \vert f(y) \vert \,dy \le \int_0^1 \vert K(x,y) \vert \,dy$ we know that $\mathcal{F}$ is pointwise bounded. Thus by the Arzela-Ascoli Theorem, we know that it is precompact.
    
    \item [4.64)] Let $\mathcal{F} = \{f \in C(X): \Vert f \Vert_u \le 1\ \text{and}\ N_\alpha(f) \le 1\}$ where
    \begin{align*}
        N_\alpha(f) = \underset{x\ne y}{\text{sup}} \frac{\vert f(x) - f(y) \vert}{\rho(x,y)^\alpha}.
    \end{align*}
    For any finite $\epsilon > 0$ let $\delta = \epsilon^{1/\alpha}$. Then 
    \begin{align*}
        \vert f(x) - f(y) \vert \le \rho(x,y)^\alpha \le \delta^\alpha = \epsilon
    \end{align*}
    implying that $\mathcal{F}$ is equicontinuous. It is also pointwise bounded by definition and so $\mathcal{F}$ is precompact by the Arzela-Azcoli Theorem.

    For any sequence $\langle f_n \rangle \subset \mathcal{F}$ convergent to some $f \in C(X)$ and for each point $x \in X$, each sequence $\langle f_n(x)\rangle$ is a convergent sequence of points in a closed set $[-1,1]$ such that $f(x) \in [-1,1]$. But this means that $\Vert f \Vert_u \le 1$. Also, since $N_\alpha(f_n) \le 1$ we have that $\vert f_n(x) - f_n(y) \vert \le \rho(x,y)^\alpha$. But $\langle f_n(x) - f_n(y) \rangle \rightarrow f(x) - f(y)$ and so it holds that $\vert f(x) - f(y) \vert \le \rho(x,y)^\alpha$. Thus $N_\alpha(f) \le 1$ and $f \in \mathcal{F}$. This means that $\overline{\mathcal{F}} = \mathcal{F}$ and so $\mathcal{F}$ is compact.

    \item [4.68)] Letting $\mathcal{F} = \{f(x,y) = g(x)h(y): g \in C(X), h \in C(Y)\}$, note that the set of finite sums of elements of $\mathcal{F}$ form an algebra. For instance, given any two elements $f,f'$ in this algebra, it holds that
    \begin{align*}
        ff' = \Big(\sum_i g_i h_i\Big)\Big(\sum_j g_j' h_j'\Big) = \sum_{i,j}(g_ig_j')(h_ih_j')
    \end{align*}
    which is still an element of $\mathcal{F}$. The algebra $\mathcal{A}$ generated by $\mathcal{F}$ contains this algebra and so they must be equivalent. Also for any $f \in \mathcal{A}$ we have that $f^* = \sum g_i^*h_i^* \in \mathcal{A}$ such that the algebra is closed under complex conjugation. This also implies that $\overline{\mathcal{A}}$ is closed under conjugation since it is a continuous function which we can apply to any convergent sequence in $\mathcal{A}$.

    Assume we have two unique points $(x_1,y_1),(x_2,y_2) \in X \times Y$ and that $x_1 \ne x_2$ without loss of generality. Then, since $X$ is compact Hausdorff and hence normal, Urysohn's Theorem tells us that there exists a function $g \in C(X)$ that separates $x_1$ and $x_2$. This in turn means that $f = g \cdot 1 \in \mathcal{A}$ separates $(x_1,y_1)$ and $(x_2,y_2)$. Thus $\overline{\mathcal{A}}$ separates points in $X \times Y$. Since $\overline{\mathcal{A}}$ also contains every constant function and since $X \times Y$ is compact, the Complex Stone-Weierstrass Theorem tells us that $\overline{\mathcal{A}} = X \times Y$, i.e. $\mathcal{A}$ is dense in $X \times Y$.
    
    \item [4.69)] Assume we have two points $x,y \in X$ and that $x \ne y$. Then there exists some coordinate map such that $\pi_\alpha(x) \ne \pi_\alpha(y)$. Thus $\mathcal{A}$ and hence it's closure separates $X$. We know that $\overline{\mathcal{A}}$ is closed under complex conjugation because the algebras of projections and the constant function are closed under conjugation and because the conjugation function is continuous. We know that $X$ is compact by Tyconoff's Theorem, and that the entire set $\mathcal{A}$ vanishes nowhere because it contains constant functions. Thus we can apply the Complex Stone-Weierstrass Theorem to show that $\mathcal{A}$ is dense in $X$.
\end{enumerate}
\end{document}