\documentclass[10pt]{article}

\usepackage{amsthm}
\usepackage{amsmath}
\usepackage{amsfonts}
\usepackage[margin=1in]{geometry}
\usepackage{titlesec}

\titleformat{\section}{\large\bfseries}
{\thesection}{0.5em}{}
\titleformat{\subsection}{\normalsize\itshape}
{\thesubsection}{0.5em}{}
\titlespacing*{\subsection}{0pt}{0pt}{0.1em}

\setlength{\parskip}{\baselineskip}
\setlength{\parindent}{0pt}

\title{Math 275A: Homework 1}
\author{Marcus Lucas}
\date{\vspace{-1cm}}


\begin{document}
\maketitle
\section*{1.1.3}
  Claim: $\mathcal{R}^d$ is countably generated.

  Proof: We know that the collection of open rectangles with rational vertices,
  $\mathcal{C}$, is countable as it can be represented by a finite produce
  of countable spaces
  \begin{equation*}
    \mathcal{C} = \{(x_{11},x_{12}, \dots, x_{1d}) \times
    \dots \times (x_{d1},x_{d2}, \dots, x_{dd}): x_{11},\dots,x_{1d}, x_{21},
    \dots,x_{dd} \in \mathbb{Q}\}
  \end{equation*} 
  It follows from this and the density of $\mathbb{Q}$ in $\mathbb{R}$
  that every open set $U \subset \mathbb{R}^d$ is a countable union of 
  such rectangles.

  In class, we showed that given two collections $\mathcal{A}$ and $\mathcal{B}$
  for which $\mathcal{A} \subset \mathcal{B}$, it follows that
  $\sigma(\mathcal{A}) \subset \sigma(\mathcal{B})$. 
  We also showed that, by definition, 
  $\sigma(\sigma(\mathcal{A})) = \sigma(\mathcal{A})$ 
  for any $\mathcal{A}$, as $\sigma(\mathcal{A})$ is the minimal $\sigma$-algebra
  containing $\mathcal{A}$.

  Now let $\mathcal{U}$ represent the collection of open sets in $\mathbb{R}^d$.
  $\mathcal{U}$ contains $\mathcal{C}$ such that,
  $\sigma(\mathcal{C}) \subset \sigma(\mathcal{U}) = \mathcal{R}^d$.
  At the same time, $\sigma(\mathcal{C})$ contains every countable union
  of sets from $\mathcal{C}$ (i.e. the open sets), such that $\mathcal{U} \subset \sigma(\mathcal{C})$.
  But this implies that 
  $\sigma(\mathcal{U}) \subset \sigma(\sigma(\mathcal{C})) = \sigma(\mathcal{C})$,
  establishing that $\sigma(\mathcal{C})$ and $\mathcal{R}^d$
  are equivalent

\section*{1.2.4}
  Claim: Show that if $F(x) = P(X \le x)$ is continuous then $Y = F(x)$ has a
  uniform distribution of $(0,1)$, the is, if $y \in [0,1], P(Y \le y) = y$.

  Proof: I take it for granted here that $F(X)$ is 
  a measurable function in the sense that $F(X)^{-1}(B) \in \mathcal{F}$
  for $B \in \mathcal{R}$ where $\mathcal{F}$ is the $\sigma$-algebra
  associated with our underlying probability space. Thus we know that
  there exists a distribution function $F'(y)$ that describes the measure
  induced on $Y$ by $F(X)$.

  Define an function $F^{-1}$ as
  \begin{equation*}
    F^{-1}(y) = \sup\{x: F(x) = y\}.
  \end{equation*}
  Note that if $F^{-1}$ were only right-continuous, it may 
  not be well-defined at some points $y$ in $Y = [0,1]$. 
  As is, evey $y$ is mapped to a unique real number by $F^{-1}$. 
  Furthermore for any given $y$, it holds that $F(F^{-1}(y)) = y$.
  
  The distribution $F'$ can be written as
  \begin{equation*}
    F'(y) = P(Y \le y) = P\{w: F(X(w)) \le y\}
      = P\{w: X(w) \le F^{-1}(y)\}.
  \end{equation*}
  To verify the last equality note that if $x$ is such that
  $F(x) \le y$ then $x$ must be less than the supremum
  of $x$ such that $F(x)=y$. Likewise if $F(x) > y$,
  then $x$ must be greater that $\sup\{x: F(x) = y\}$.

  By the definition of the distribution function of $X$,
  it follows that 
  \begin{equation*}
    F'(y) = F(F^{-1}(y)) = y.
  \end{equation*}

\section*{1.3.1}
  Claim: If $\mathcal{A}$ generates $\mathcal{S}$, 
  then $X^{-1}(\mathcal{A}) = \{\{X \in A\} : A \in \mathcal{A}\}$
  generates $\sigma(X) = \{\{X \in B\} : B \in \mathcal{S}\}$.

  Proof: By the fact that $\mathcal{A}$ is a subcollection of sets
  in $\mathcal{S}$, we know that $X^{-1}(\mathcal{A}) \subset \sigma(X)$,
  from which it follows that $\sigma(X^{-1}(\mathcal{A})) \subset \sigma(X)$.
  To show the reverse inclusion, first note two equalities:
  \begin{equation*}
    \begin{aligned}
      \{X \in \cup_i A_i\} & = \cup_i \{X \in A_i\} \\
      \{X \in A^c\} & = \{X \in A\}^c.
    \end{aligned}
  \end{equation*}
  The equalities imply that $\sigma(X^{-1}(\mathcal{A}))
  = \{\{X \in A\} : A \in \sigma(\mathcal{A})\}$.
  The prior inclusion $\sigma(X^{-1}(\mathcal{A})) \subset \sigma(X)$
  then becomes an equality and so
  $X^{-1}(\mathcal{A})$ must generate $\sigma(X)$.

\section*{1.3.4}
  Claim: (i) Show that a continuous function from $\mathbb{R}^d \to \mathbb{R}$
  is a measurable map from $(\mathbb{R}^d, \mathcal{R}^d)$.
  (ii) Show that $\mathcal{R}^d$ is the smallest $\sigma$-field
  that makes all the continuous functions measurable.

  Proof: (i) If $f : \mathbb{R}^d \to \mathbb{R}$ 
  is continuous then its inverse $f^{-1}(A)$ 
  is open for any open set $A$.
  Let $\mathcal{A}$ represent the open sets in $\mathbb{R}$ which
  generate $\mathcal{R}$. Then all the sets in $f^{-1}(\mathcal{A})$
  are open and members of $\mathcal{R}^d$.
  Applying Theorem 1.3.1, we can conclude that $f$
  must be measurable.
  
  (ii) From (i) we can conclude that $\mathcal{R}^d$ is a sufficiently
  rich collection of sets so as to make any continuous function 
  $f: \mathbb{R}^d \to \mathbb{R}$ measurable. To show that it is also 
  the smallest such $\sigma$-field, note that any candidate $\sigma$-field
  $\mathcal{F}$ must contain $f^{-1}(\mathcal{A})$ for every continuous
  function $f$.

  Take the set of $d$ projection functions $\pi_i$ that map each 
  vector in $x \in \mathbb{R}^d$ onto its $i$-th coordinate $x_i$.
  Let $\mathcal{A}_i = \pi_i^{-1}(\mathcal{A})$. Every open
  set in $\mathbb{R}^d$ of the form $U_1 \times \dots \times U_d$,
  for $U_i \in \mathbb{R}$, can be described as some intersection 
  $\cap_{i=1}^d A_i$ of sets where $A_i \in \mathcal{A}_i$.
  The collection of such sets defines a basis for the standard topology
  on $\mathbb{R}^d$. Thus any $\sigma$-field that renders all 
  continuous functions measurable must contain $\mathcal{R}^d$,
  as it will contain at least all the open sets in $\mathbb{R}^n$.
  So then $\mathcal{R}^d$ itself must be the smallest
  $\sigma$-algebra that reders such functions measureable.

\section*{Bonus}
  Claim: Let $(\Omega, \mathcal{F}, \mu)$ be a measure space and let
  $E \in \mathcal{F}$. Define:
  \begin{equation*}
    \mu_E(A) := \mu(A \cap E) \quad \text{for} \quad A \in \mathcal{F}
  \end{equation*}
  Prove that $\mu_E$ is also a measure on $(\Omega, \mathcal{F})$

  Proof: For any set $A \in \mathcal{F}$, $A \cap E$
  is also a set in $\mathcal{F}$, it being a $\sigma$-algebra.
  This means that $\mu_E$ is well-defined for all sets $A \in \mathcal{F}$
  and that it must satisfy the non-negativity and null empty set
  properties of a measure.

  As for countable additivity, 
  given a countable collection of disjoint sets $\{A_i\}_i$,
  \begin{equation*}
    \mu_E(\cup_i A_i) = \mu(\cup_i A_i \cap E) = \cup_i \mu(A_i \cap E)
    = \cup_i \mu_E(A_i)
  \end{equation*}
  where in the second equality we exploit the fact that the sets
  $\{A_i \cap E\}_i$ must also be disjoint. 
  Thus countable additivity holds as well.

\end{document}