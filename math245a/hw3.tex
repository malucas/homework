\documentclass[11pt,letter]{article}
\usepackage{fancyhdr}
\usepackage[]{amsmath}
\usepackage[margin=1in]{geometry}
\usepackage{amsfonts}
\usepackage{amssymb}
\pagestyle{fancy}
\fancyhead[LO]{MATH245A: HW \#3}
\fancyhead[CO]{Marcus Lucas}
\fancyhead[RO]{October 23, 2018}

\begin{document}
\begin{enumerate}
    \item [2.6] \textbf{Claim:} The supremum of an uncountable family of measurable $\overline{\mathbb{R}}$-valued functions $X$ can fail to be measurable.
    
    \textbf{Proof:} Define 
    \begin{align*}
        g = \underset{\alpha \in \mathcal{A}}{\text{ sup }} f_\alpha(x)
    \end{align*} 
    where $\mathcal{A}$ is an uncountable index set. Then we have that
    \begin{align*}
        g^{-1} ((a,\infty]) = \bigcup_{\mathcal{A}} f_\alpha^{-1} ((a,\infty])
    \end{align*}
    but in general, $f_\alpha^{-1} ((a,\infty]) \in \mathcal{M}$ and $\mathcal{M}$, being a $\sigma$-algebra, is not necessarily closed under uncountable unions. For instance, if $\mathcal{M} = \mathcal{B}_{\mathbb{R}}$ and $f_\alpha^{-1} ((a,\infty])$ is a closed set for each $\alpha$, then $g^{-1}((a,\infty])$ is an uncountable union of closed sets and so not in $\mathcal{B}_{\mathbb{R}}$. Thus, in general, $g$ as defined above is not measurable.

    \item [2.7] \textbf{Claim:} Suppose that for each $\alpha \in \mathbb{R}$ we are given a set $E_\alpha \in \mathcal{M}$ such that $E_\alpha \subset E_\beta$ whenever $\alpha < \beta$, $\bigcup_{\alpha \in \mathbb{R}} E_\alpha = X$, and $\bigcap_{\alpha \in \mathbb{R}} E_\alpha = \emptyset$. Then there is a measurable function $f: X \rightarrow \mathbb{R}$ such that $f(x) \le \alpha$ on $E_\alpha$ and $f(x) \ge \alpha$ on $E_\alpha^c$ for every $\alpha$.
    
    \textbf{Proof:} Let $f(x) = \text{inf }\{q \in \mathbb{Q}:x \in E_q\}$. By definition $f^{-1}((q,\infty]) = E_q \in \mathcal{M}$ for any $q \in \mathbb{Q}$. By the result of exercise 2.4, this tells us that $f$ is measurable.

    Given some $x \in E_\alpha$ for any $\alpha \in \mathbb{R}$, it holds that $x \in E_q$ for every rational $q \ge \alpha$. But $f(x) \le q$ for all such $q$ and, as $\mathbb{Q}$ is dense in $\mathbb{R}$, $f(x) \le \alpha$ must also hold. Likewise, for any $q' \le \alpha$, $x \in E_a^c$ implies that $x \in E_{q'}^c$. Thus $f(x) > q'$ for all $q' \le \alpha$ implying $f(x) \ge \alpha$ on $E_\alpha^c$.

    \item [2.8] \textbf{Claim:} If $f:\mathbb{R} \rightarrow \mathbb{R}$ is monotone, then $f$ is Borel measurable.
    
    \textbf{Proof:} By Propositions 2.6 we know that if $f$ is measurable then $-f$ is measurable, so assume that $f$ is increasing. 
    
    Take a finite open interval $(a,b)$. For any $x \in f^{-1}((a,b))$, $a < f(x) < b$ holds, meaning we can always find $c,d$ such that $a < c < f(x) < d < b$. But then $f^{-1}(a) < f^{-1}(c) < x < f^{-1}(d) < f^{-1}(b)$ by the monotonicity of $f$. So then for any $x \in f^{-1}((a,b))$ we can always find some open interval $(f^{-1}(c), f^{-1}(d)) \subset f^{-1}((a,b))$ which contains $x$. Thus $f^{-1}((a,b))$ is an open set and an element of $\mathcal{B}_{\mathbb{R}}$. Since $\mathcal{B}_{\mathbb{R}}$ is generated by the set of all open intervals on $\mathbb{R}$, this means that $f$ is measurable.

    \item [2.11] \textbf{Claim:} Suppose $f$ is a function on $\mathbb{R} \times \mathbb{R}^k$ such that $f(x,\cdot)$ is Borel measurable for each $x \in \mathbb{R}$ and $f(\cdot,y)$ is continuous for each $y \in \mathbb{R}^k$. For $n \in \mathbb{N}$, define $f_n$ as follows. For $i \in \mathbb{Z}$ let $a_i = i/n$, and for $a_i \le x \le a_{i+1}$ let
    \begin{align*}
        f_n(x,y) = \frac{f(a_{i+1},y)(x-a_i) - f(a_i,y)(x - a_{i+1})}{a_{i+1} - a_i}
    \end{align*}
    Then $f_n$ is Borel measurable on $\mathbb{R} \times \mathbb{R}^k$ and $f_n \rightarrow f$ pointwise; hence $f$ is Borel measurable on $\mathbb{R} \times \mathbb{R}^k$. Also, every function on $\mathbb{R}^n$ that is continuous in each variable separately is Borel measurable.

    \textbf{Proof:} $f(\cdot, y)$ is continuous function between metric spaces and so is Borel measurable. Thus $f_n(x,y)$ is Borel measurable on any $[a_i,a_{i+1}]$ as it is the sum and product of Borel measurable functions. In general, this means that for any $E \in \mathcal{B}_{\mathbb{R}}$, the intersection $f_n^{-1}(E) \cap [a_i,a_{i+1}]$ is Borel measurable. But $f_n^{-1}(E) = \bigcup_{i \in \mathbb{Z}} \big(E \cap [a_i,a_i+1] \big)$, which is just a countable union of Borel sets, and so $f_n^{-1}(E)$ is Borel measurable on its entire domain.

    For any $\epsilon > 0$ we can find $\delta > 0$ (for a point $x \in \mathbb{R}$) such that $|f(x', y) - f(x,y)| < \epsilon$ whenever $|x' - x| < \delta$. For any delta, we can also find an $N \in \mathbb{N}$ such that $\frac{1}{n} < \delta$ for all $n > N$. Thus at any point $x \in [a_i, a_{i+1}]$, with fixed $\epsilon, \delta$, and $N$ such that $a_i,a_{i+1} \in (x-\delta, x+\delta)$ for all $n > N$, we have
    \begin{align*}
        |f_n(x,y) - f(x,y)| & = \Big| \frac{f(a_{i+1},y)(x-a_i) - f(a_i,y)(x - a_{i+1}) - (a_{i+1} - a_i)f(x,y)}{a_{i+1} - a_i} \Big| \\
        & = \Big| \frac{f(a_{i+1},y)(x-a_i) - f(a_i,y)(x - a_{i+1}) - (a_{i+1} -x + x - a_i)f(x,y)}{a_{i+1} - a_i} \Big| \\
        & = \Big| \frac{\big( f(a_{i+1},y) - f(x,y)\big) (x-a_i) - \big( f(a_i,y) - f(x,y)\big) (x - a_{i+1}) }{a_{i+1} - a_i} \Big| \\
        & \le \Big| \frac{\big( f(a_{i+1},y) - f(x,y)\big) (x-a_i)}{a_{i+1} - a_i} \Big| - \Big| \frac{\big( f(a_i,y) + f(x,y)\big) (x - a_{i+1}) }{a_{i+1} - a_i} \Big| \\
        & \le \frac{\epsilon (x-a_i)}{a_{i+1} - a_i} + \frac{\epsilon (a_{i+1} - x)}{a_{i+1} - a_i} = \epsilon
    \end{align*}
    So for any $\epsilon$ we can find $N$ such that $|f_n(x,y) - f(x,y)| < \epsilon$ and thus $f_n$ converges pointwise to $f$. This along with Proposition 2.7 implies that $f$ is Borel measurable.

    Finally, we know that functions on $\mathbb{R}$ that are continuous are Borel measurable. Now assume that every function on $\mathbb{R}^k$ which is continuous in each variable separately is also Borel measurable. If we're given a function $f$ on $\mathbb{R}^{k+1} = \mathbb{R} \times \mathbb{R}^{k}$ that is continuous in each variable separately, we know that $f(\cdot,y): \mathbb{R} \rightarrow \mathbb{R}$ is a continuous function and hence Borel measurable. Also, $f(x,\cdot): \mathbb{R}^k \rightarrow \mathbb{R}$ is continuous in each variable separately and so, by our assumption, it is Borel measurable. As we've just proved, this means that $f$ is Borel measurable on $\mathbb{R}^{K=+1}$. So, in general, functions on any finite dimensional space $\mathbb{R}^n$ which are continous in each variable separately are also Borel measurable.

    \item [2.14] \textbf{Claim:} If $f \in L^+$, let $\lambda(E) = \int_E f\,d\mu$ for $E \in \mathcal{M}$. Then $\lambda$ is a measure on $\mathcal{M}$, and for any $g \in L^+$, $\int g\,d\lambda = \int fg\,d\mu$.

    \textbf{Proof:} Theorem 2.10 tells us that we can construct a sequence $\{\phi_n\}$ of increasing simple functions which converge pointwise to $f$. For any disjoint sequence $\{E_j\} \subset \mathcal{M}$ such that $\bigcup_j{E_j} = E$ we have
    \begin{align*}
        \int_E f = \underset{n \rightarrow \infty}{\text{ lim }} \int_E \phi_n
    \end{align*}
    by the monotone convergence theorem. By thereom 2.13(d) we already know that restrictions of integrals of simple functions to sets in $\mathcal{M}$ define measures on $\mathcal{M}$. Thus
    \begin{align*}
        \underset{n \rightarrow \infty}{\text{ lim }} \int_E \phi_n = \underset{n \rightarrow \infty}{\text{ lim }} \sum_j \int_{E_j} \phi_n =  \sum_j \underset{n \rightarrow \infty}{\text{ lim }} \int_{E_j} \phi_n 
    \end{align*}
    Again by the monotone convergence theorem, the simple functions $\phi_n$ are still increasing and approaching $f$ when restricted to $E_j$, so
    \begin{align*}
        \sum_j \underset{n \rightarrow \infty}{\text{ lim }} \int_{E_j} \phi_n  = \sum_j \int_{E_j} f
    \end{align*}
    and so
    \begin{align*}
        \lambda(E) = \int_E f = \sum_j \int_{E_j} f = \sum_j \lambda(E_j)
    \end{align*}
    establishing that $\lambda$ is a measure on $\mathcal{M}$. $_\square$

    Now suppose we have a simple function $g \in L^+$ defined over a collection of measures $\{E_j\}_1^N$. Then
    \begin{align*}
        \int g\,\lambda = \sum_j a_j \lambda(E_j) = \sum_j a_j \int_{E_j} f\, d\mu = \sum_j \int a_j f\chi_{E_j}\, d\mu = \int f \sum_j a_j \chi_{E_j}\, d\mu = \int f g\,d\mu
    \end{align*}

    More generally, if $g$ is any function in $L^+$, prop 2.10 tells us that we can create a sequence of increasing simple functions $\{\psi_n\}$ that approach $g$ pointwise. Therefore $\{f \psi_n\}$ approaches $fg$ pointwise and we can apply the monotone convergence theorem twice to get
    \begin{align*}
        \int g\,d\lambda = \underset{n \rightarrow \infty}{\text{ lim }} \int \psi_n\,\lambda = \underset{n \rightarrow \infty}{\text{ lim }} \int f\psi_n\,\mu = \int fg\,d\mu \quad _\square 
    \end{align*}

    \item [2.16] \textbf{Claim:} If $f \in L^+$ and $\int f < \infty$, for every $\epsilon > 0$ there exists $E \in \mathcal{M}$ such that $\mu(E) < \infty$ and $\int_E f > (\int f) - \epsilon$.
    
    \textbf{Proof:} We know by proposition 2.10 that there exists a series of increasing simple functions $\{\phi_n\}$ whice approach $f$ pointwise and that (by the MCT) 
    \begin{align*}
        \int f = \underset{n \rightarrow \infty}{\text{ lim }} \int \phi_n
    \end{align*}
    Thus for any $\epsilon > 0$, there exists a simple function $\phi$ such that $0 \le \phi \le f$ and
    \begin{align*}
        \int \phi > (\int f) - \epsilon
    \end{align*}
     But each $\phi = \sum_j a_j \chi_{E_j}$ for some finite collection of sets $\{E_j\}_1^N \subset \mathcal{M}$. This being the standard representation of $\phi$, only one $a_j$, let's assume $a_1$ can equal zero. Furthermore, $\int \phi < \infty$ implies that $E = E_2 \cup \cdots \cup E_N$ must have finite measure since 
     \begin{align*}
        \int \phi = \sum_{j=2}^N a_j \mu(E_j) \ge \text{ min}\{a_j\}_2^N \mu(E)
     \end{align*} 
     So $E$ is a set such that $E \in \mathcal{M}$, $\mu(E) < \infty$, and
     \begin{align*}
         \int_E f \ge \int_E \phi = \int \phi \ge (\int f) - \epsilon
     \end{align*}

\end{enumerate}
\end{document}